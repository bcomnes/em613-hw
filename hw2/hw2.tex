\documentclass{jhwhw}
\usepackage{color}
\usepackage{amsmath}
\usepackage{graphicx}
\usepackage{hyperref}
\usepackage{braket}
\usepackage{cancel}
%\everymath{\displaystyle}
%\usepackage{bm}
%\usepackage{setspace}
%\usepackage{verbatim}
%\usepackage[lmargin=2.5cm, rmargin=2.5cm,tmargin=3cm,bmargin=2.5cm]{geometry}
%
% Hyperlink styling
\hypersetup{
colorlinks=true,
linkcolor=blue,
urlcolor=blue
}
%
% This allows you to draw a box around things.
% Wrap whateve you want a box around with:
%\Aboxed{Stuff you want boxed}
\def\@Aboxed#1&#2\ENDDNE{%
  \settowidth\@tempdima{$\displaystyle#1{}$}%
  \addtolength\@tempdima{\fboxsep}%
  \addtolength\@tempdima{\fboxrule}%
  \global\@tempdima=\@tempdima
  \kern\@tempdima
  &
  \kern-\@tempdima
  \fcolorbox{red}{yellow}{$\displaystyle #1#2$}
}
%
% Part of AMS Math
% Lets you create a log-like trace function
\DeclareMathOperator{\Tr}{Tr}
%
% Author and title
\author{Bret Comnes}
\title{E\&M 613 Homework 2}

\begin{document}
\problem{J: 1.1(c)}
Use Gauss's theorem to prove that the electric field at the surface of a conductor is normal to the surface and has a maganitude of $4 \pi k_E \sigma$, where $\sigma$ is the charge density per unit area on the surface.

\solution
Starting with Gauss's Law, we can work out the volume integral to solve one side of the equation.  This volume integral should incorporate all of $\vec{E}$, including any possible tangential components.

\begin{align}
    \label{eq:gauss}\int_V  \vec{\nabla} \cdot \vec{E} d\tau &= \oint_S \vec{E} \cdot d\vec{a}
\end{align}

For a continuous, volumetric charge distribution

\begin{align}
    \label{eq:evol}\vec{E}(r) &= k_e \int_V \frac{\rho(\vec{r'})}{R^2}\hat{R}d\tau'
\end{align}

Putting \eqref{eq:evol} into Equation~\eqref{eq:gauss}

\begin{align}
    \int_V  \vec{\nabla} \cdot \left(k_e \int_V \frac{\rho(\vec{r'})}{R^2}\hat{R}d\tau'\right) d\tau &= \oint_S \vec{E} \cdot d\vec{a} \\
    \int_V  \left(k_e \int_V \vec{\nabla} \cdot \frac{\hat{R}}{R^2}\rho(\vec{r'})d\tau'\right) d\tau &= \oint_S \vec{E} \cdot d\vec{a}
\end{align}

Remembering that

\begin{align}
    \vec{\nabla} \cdot \left( \frac{\hat{R}}{R^2} \right) &= 4 \pi \delta^3(\vec{R})
\end{align}

Thus

\begin{align}
    \int_V  
    \left(k_e 
    \int_V 4 \pi \delta^3(\vec{r}-\vec{r'}) \rho(\vec{r'})d\tau'
    \right) d\tau
    &= \oint_S \vec{E} \cdot d\vec{a} \\
    \label{eq:volint}
    4\pi k_e 
    \int_V  
    \rho d\tau
    &=
    \oint_S \vec{E} \cdot d\vec{a}
\end{align}

\pagebreak[3]

Now we can focus on the RHS of this equation.  Lets draw a Gaussian, cylindrical pillbox around the surface of our boundary.

\vspace{50 mm}

As we reduce the height of this pillbox to nothing, a couple of important things happen.  Our volume integral from Equation~\eqref{eq:volint} reduces to an area integral and our volumetric charge density, $\rho$ reduces to $\sigma$.  Also, our surface integral in the radially normal direction of our cylindrical pillbox disappears.

\begin{align}
    4\pi k_e 
    \int_V  
    \sigma da
    &=
    \oint_S \vec{E_1} \cdot \hat{n_1} da + 
    \cancelto{0}{\oint_S \vec{E}_{\hat{r}} \cdot \hat{n}_{\hat{r}} da} + 
    \oint_S \vec{E_2} \cdot \hat{n_2} da
\end{align}

When you dot a vector into a unit vector (in this case, normal unit vectors), you get the magnitude of that vector in the unit vector direction.  Since region one points down, we give it a negative value and region 2 points up so it gets a positive value.  Since region one is a perfect conductor, $E_{2}$ goes to 0.

\begin{align}
    4\pi k_e  
    \sigma
    &=
    E_{2\hat{n}} - \cancelto{0}{E_{1\hat{n}}}
    \\
    \Aboxed{
    4\pi k_e  
    \sigma
    &=
    E_{2\hat{n}}
    }
\end{align}

Because our volume integral incorporates all of the electric field near the surface, and our surface integrals only take into account the normal components of the electric field, and they both satisfy Gauss's law, we can conclude that there are no tangential components to the electric field at the surface.

\problem{J: 1.3(a,c)}
Using the Dirac delta functions in the appropriate coordinates, express the following charge distributions as three dimensional charge densities $\rho{(\vec{r})}$.

\part
In spherical coordinates, a charge $Q$ uniformly distributed over a spherical shell of radius $R$.

\pagebreak[4]

\part
In cylindrical coordinates, a charge of $Q$ spread uniformly over a flat circular disc of negligible thickness and radius $R$.

\problem{J: 1.5}
The time averaged potential of a neutral hydrogen atom is given by:
\begin{align}
    \Phi(r) &= k_E q \frac{e^{-\alpha r}}{r} \left(1 + \frac{\alpha r}{2} \right)
\end{align}
where $q$ is the magnitude of the electronic charge, and $\alpha^-1 = a_0 / 2$, $a_0$ being the Bohr radius.  Find the distribution of charge (both continuously and discrete) which will give this potential and interpret your result physically.

\problem{}
Two charges, $q$ and $q'$ are located, respectively, inside and outside a hollow conductor.  Charge $q'$ experiences a force due to $q$, but not vice versa.  Prove this statement and explain this apparent violation of Newton's third law.  There is no net charge on the conductor.  

\problem{Earnshaw's Theorem:}
Use Laplace's equation to show that a charged body placed in an electric field cannon be maintained in a position of stable equilibrium by the application of electrostatic forces alone.

\problem{Thomson's theorem (J: 1.15)}
Prove that for a number of conduction surfaces fixed in position and a given total charge is placed on each surface, then the electrostatic energy in the region bounded by the surfaces is an absolute minimum when the charges are placed so that every surface is an equipotential, as happens when they are conductors.

\problem{}
The mutual capacitance (C) of two conductors carrying charges $Q$ and $-Q$ is defined through $Q = C (V_2 - V_1)$, where $V$'s are the potentials on each of them.  Show the $C$ could be given in terms of the coefficients of capacity and those of electrostatic induction in the form:
\begin{align}
    C &= \frac{C_{11}C_{22} - C^2_{12}}{C_{11}+2C_{12}+C_{22}}
\end{align}

\end{document}