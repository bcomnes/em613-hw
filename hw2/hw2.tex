\documentclass{jhwhw}
\usepackage{color}
\usepackage{amsmath}
\usepackage{graphicx}
\usepackage{hyperref}
\usepackage{braket}
\usepackage{cancel}
%\everymath{\displaystyle}
%\usepackage{bm}
%\usepackage{setspace}
%\usepackage{verbatim}
%\usepackage[lmargin=2.5cm, rmargin=2.5cm,tmargin=3cm,bmargin=2.5cm]{geometry}
%
% Hyperlink styling
\hypersetup{
colorlinks=true,
linkcolor=blue,
urlcolor=blue
}
%
% This allows you to draw a box around things.
% Wrap whateve you want a box around with:
%\Aboxed{Stuff you want boxed}
\def\@Aboxed#1&#2\ENDDNE{%
  \settowidth\@tempdima{$\displaystyle#1{}$}%
  \addtolength\@tempdima{\fboxsep}%
  \addtolength\@tempdima{\fboxrule}%
  \global\@tempdima=\@tempdima
  \kern\@tempdima
  &
  \kern-\@tempdima
  \fcolorbox{red}{yellow}{$\displaystyle #1#2$}
}
%
% Part of AMS Math
% Lets you create a log-like trace function
\DeclareMathOperator{\Tr}{Tr}
%
% Author and title
\author{Bret Comnes}
\title{E\&M 613 Homework 2}

\begin{document}
\problem{J: 1.1(c)}
Use Gauss's theorem to prove that the electric field at the surface of a conductor is normal to the surface and has a maganitude of $4 \pi k_E \sigma$, where $\sigma$ is the charge density per unit area on the surface.

\solution
Solution

\problem{J: 1.3(a,c)}
Using the dirac delta functions in the appropriate coordinates, express the following charge distributions as three dimensional charge densities $\rho{(\vec{r})}$.

\part
In spherical coordinates, a charge $Q$ uniformly distributed over a spherical shell of radius $R$.

\pagebreak[4]

\part
In cylindrical coordinates, a charge of $Q$ spread uniformly over a flat circular disc of negligible thickness and radius $R$.

\problem{J: 1.5}
The time averaged potential of a neutral hydrogen atom is given by:
\begin{align}
    \Phi(r) &= k_E q \frac{e^{-\alpha r}}{r} \left(1 + \frac{\alpha r}{2} \right)
\end{align}
where $q$ is the magnitude of the electronic charge, and $\alpha^-1 = a_0 / 2$, $a_0$ being the Bohr radius.  Find the distribution of charge (both continuously and discrete) which will give this potential and interpret your result physically.

\problem{}
Two charges, $q$ and $q'$ are located, respectively, inside and outside a hollow conductor.  Charge $q'$ experiences a force due to $q$, but not vice versa.  Prove this statement and explain this apparent violation of Newton's third law.  There is no net charge on the conductor.  

\problem{Earnshaw's Theorem:}
Use Laplace's equation to show that a charged body placed in an electric field cannon be maintained in a position of stable equilibrium by the application of electrostatic forces alone.

\problem{Thomson's theorem (J: 1.15)}
Prove that for a number of conduction surfaces fixed in position and a given total charge is placed on each surface, then the electrostatic energy in the region bounded by the surfaces is an absolute minimum when the charges are placed so that every surface is an equipotential, as happens when they are conductors.

\problem{}
The mutual capacitance (C) of two conductors carrying charges $Q$ and $-Q$ is defined through $Q = C (V_2 - V_1)$, where $V$'s are the potentials on each of them.  Show the $C$ could be given in terms of the coefficients of capacity and those of electrostatic induction in the form:
\begin{align}
    C &= \frac{C_{11}C_{22} - C^2_{12}}{C_{11}+2C_{12}+C_{22}}
\end{align}

\end{document}